\documentclass[12pt]{article}
% Basic packages
\usepackage[utf8]{inputenc}   % UTF-8 encoding
\usepackage{geometry}         % Page margins
\usepackage{setspace}         % Double spacing
\usepackage{amsmath, amssymb} % Math symbols
\usepackage{graphicx}         % Include images
\usepackage{booktabs}         % Nice tables
\usepackage{enumitem}         % Custom lists
\usepackage{cancel}
\usepackage{graphicx}
\usepackage{float}

% Page setup: 1-inch margins
\geometry{
  letterpaper,
  margin=1in
}

% Double spacingp
\doublespacing{}

% START OF DOCUMENT %
\begin{document}

\begin{center}
    {\textbf{Johnny A. Rojas \& Aiden Sitorus, George Nicacio, Jeremiah
        Hasibuan, Matthew Miller}}\\
    {\Large \textbf{LAB REPORT: ELECTRIC FIELD AND EQUIPOTENTIALS}}
\end{center}

\section{Theory}
\subsection*{From Vector Fields to Electric Potential}

In physics, vector fields are mathematical tools that allow us to visualize how
a physical quantity with both magnitude and direction behaves throughout space.
Rather than describing a single interaction, a field describes how one object
influences every possible point in its surroundings. This perspective allows a
single equation to encode the structure of a physical interaction in space.

A natural starting point for understanding electric fields is gravitation.
Two objects with mass attract each other with a force proportional to the
product of their masses and inversely proportional to the square of the
distance between their centers of mass:

\begin{align*}
F_G = G \frac{M m}{R^2}
\end{align*}

where $G$ is the universal gravitational constant.

The inverse–square dependence is not accidental. As distance increases,
the interaction spreads over larger spherical surfaces, producing the
$1/R^2$ behavior. This geometric interpretation will later help us understand
electric fields.

\subsubsection*{From Force to Gravitational Field}

Instead of focusing only on the interaction between two masses, we divide
the force by one of the masses. This removes dependence on the test mass
and isolates a quantity that depends only on the source mass and position:

\begin{align*}
g = \frac{F_G}{m} = G \frac{M}{R^2}
\end{align*}

The quantity $g$ is the gravitational field — a vector field describing
acceleration due to gravity at every point in space.

At the Earth's surface:

\begin{align*}
g_0 = G \frac{M_{\text{Earth}}}{R_{\text{Earth}}^2}
\end{align*}

At altitude $H$, where $R = R_{\text{Earth}} + H$:

\begin{align*}
g = G \frac{M_{\text{Earth}}}{(R_{\text{Earth}} + H)^2}
\end{align*}

Once the field is known, the force follows immediately:

\begin{align*}
F_G = m g
\end{align*}

\subsubsection*{Gravitational Potential Energy}

Because gravity acts at a distance, an object placed in the field possesses
gravitational potential energy.

With respect to the Earth's surface:

\begin{align*}
GPE = m g H
\end{align*}

With respect to the Earth's center:

\begin{align*}
GPE = m g (R_{\text{Earth}} + H) = m g R
\end{align*}

Since $F_G = m g$, we may also write:

\begin{align*}
GPE = F_G R
\end{align*}

We now see the mathematical progression
$F_G \rightarrow g \rightarrow GPE$:
force leads to field, and field leads to potential energy.

\subsection*{The Electrical Analogy}

Charge particles also interact through a force at distance described by
Coulomb’s Law:

\begin{align*}
F_E = \frac{1}{4\pi\epsilon_0} \frac{q_1 q_2}{r^2}
\end{align*}

or equivalently,

\begin{align*}
F_E = k \frac{Q q}{R^2}
\end{align*}

where $k \approx 9 \times 10^9 \, \text{N·m}^2/\text{C}^2$ in vacuum.

Structurally, this mirrors gravitation:
$(Q, q) \leftrightarrow (M, m)$ and $k \leftrightarrow G$.
Both forces obey an inverse–square law, although electric forces may be
attractive or repulsive depending on the signs of the charges.

\subsubsection*{Vector Form and the Electric Field}

To fully describe the interaction, we introduce direction using the unit
vector $\hat{r}$:

\begin{align*}
\vec{F}_E =
\frac{1}{4\pi\epsilon_0}
\frac{q_1 q_2}{r^2} \hat{r}
\end{align*}

An important conceptual shift occurs when we divide the force by one of the
charges. In doing so, we separate the mutual interaction from the influence
of the source charge. Dividing by the test charge $q$:

\begin{align*}
\vec{E} = \frac{\vec{F}_E}{q}
= \frac{1}{4\pi\epsilon_0} \frac{Q}{r^2} \hat{r}
\end{align*}

This defines the electric field $\vec{E}$.

Here:
\begin{itemize}
  \item $\vec{E}$ has units of N/C,
  \item $Q$ is the source charge,
  \item $r$ is the distance from the source,
  \item $\hat{r}$ determines direction.
\end{itemize}

This reflects the earlier gravitational reasoning: just as $g$ describes the
influence of a mass independently of the test mass, $\vec{E}$ describes the
influence of a charge independently of any test charge.

The direction of $\vec{E}$ is defined as the direction a positive test charge
would move. Electric field lines diverge from positive charges and converge
toward negative charges. Unlike gravity, the electric interaction can be
repulsive as well as attractive.

\subsubsection*{Electric Potential Energy and Electric Potential}

Continuing the analogy, electric potential energy is defined as

\begin{align*}
EPE = F_E R
\end{align*}

Substituting Coulomb’s law:

\begin{align*}
EPE = k \frac{Q q}{R}
\end{align*}

Since $E = k Q / R^2$, we may also write:

\begin{align*}
EPE = q E R
\end{align*}

Another fundamental electrostatic relation states

\begin{align*}
EPE = q V
\end{align*}

where $V$ is the electric potential (voltage), measured in volts.

This leads to another expression for the electric field:

\begin{align*}
E = -\frac{\Delta V}{\Delta x}
\end{align*}

This equation shows that the electric field is the negative gradient of the
electric potential. Therefore, the electric field has another equivalent unit:

$1 \ \text{N/C} = 1 \ \text{V/m}$.

Conceptually, the electric field describes how rapidly electric potential
changes in space. It points in the direction of steepest decrease of
potential, analogous to the slope of a hill.

\subsection*{Connection to the Experiment}

In the laboratory, extended charge distributions are modeled using electrodes.
Regions under the same electric potential are called equipotential regions.
If $\Delta V \neq 0$, then $E \neq 0$.

Rather than measuring force directly, we measure potential differences
between nearby points using a multimeter. From these measurements, we
calculate the electric field using

\begin{align*}
E = -\frac{\Delta V}{\Delta x}
\end{align*}

Thus the experiment completes the conceptual path:

$F_E \rightarrow \vec{E} \rightarrow EPE \rightarrow V$.

We began with inverse–square forces and arrived at a measurable scalar
quantity whose spatial variation generates a vector field.

\section{Conclusion}

\subsection*{Activity 1}
Our results in Activity 1 demonstrated that the data follows a curved behavior,
increasing significantly near the electrodes and remaining close to zero near
the center. This matches our theoretical expectation for two opposite charges,
where the electric field is strongest close to the charges and weaker near the
midpoint due to symmetry. The non-linear shape of the graph is consistent with
the dipole-like configuration predicted by theory.

\subsection*{Activity 2}
Our results in Activity 2 demonstrated that the data follows the prediction of
the magnitude being approximately constant regardless of the position at which
the electric field is measured. We measured a slope of $-0.0036$, which compared
to the expected slope of $0$, gives an absolute error of $0.0036$. The $R^2$
value of $0.0004$ further supports that there is no significant linear
dependence on position. This agrees with the theoretical prediction of a nearly
uniform electric field between parallel electrodes.

\subsection*{Activity 3}
In Activity 3 we observed a strong agreement between our measurements and the
theoretical cylindrical relationship $E \propto \frac{1}{r}$. From our power
fit, we measured an exponent of $-1.01$. Compared to the expected value of $-1$,
this corresponds to an absolute error of $0.01$, or approximately $1\%$ error.
The $R^2$ value of $0.8841$ indicates a strong correlation with the predicted
inverse radial dependence. This confirms the expected cylindrical symmetry of
the electric field in this configuration.

Small differences between theoretical and experimental values are likely due to
systematic errors such as probe positioning, finite electrode size, non-ideal
conductivity of the carbon paper, and slight asymmetries in the setup.


% Key: final results of each activities: slope, power, coefficient, etc. Do not
% report any graph or chart!!! was the experiment successful or not?

% Must:

% - Report the values or numbers that were MEASURED, their expected or known
%   values if applicable, and their absolute or/and percent errors, R^2,
%   Standard deviations, standard errors if applicable. Saying "the experiment was
%   successful with 5% error" is not acceptable because it misses the measured
%   value and the expected value. It should have been "Our measured power was
%   0.475 whereas the expected power is 0.5. The experiment was successful with 5%
%   error."

% - Discuss the kind of errors: systematic or random and explain. Here, you can
%   refer to the charts (without reporting the charts). You can say that most of
%   the measured values were below the expected for example.

% - List out the source or errors and be specific. Do not just say "the frictions
%   created errors", rather say "the frictions between the wheels and the axles
%   created errors".

% - Then, state the biggest source of errors and how it could be avoided.

% - Address the outliers if any, Also, explain why you rejected a data if you did.

% - You may suggest a way to improve the experiment. That will be always
%   appreciated.

% Strategy: do the above for each activity as one activity might be successful but
% the other had an issue, then end with an overall evaluation.

\section{Questions}
% Key: they are in khpContent

% Must: Itemize the answers. For example:

% - Activity #1, question 2-a

% - Activity #3, question 3

% Strategy: Most often, you find these questions in interrogative forms ending
% with a question mark "?" But sometimes the questions are in indirect forms so
% be watchful. Never write "See Excel file", just make a copy of the Excel file
% and report it again.
\subsection*{Activity 1}

\begin{figure}[H]
  \centering
  \includegraphics[width=0.6\textwidth]{Tables/table1.png}
  \caption{Table 1.}
\end{figure}

\begin{description}

\item[Question 1.a:] How does the graph look?

        The graph of $\Delta V$ versus position is nonlinear. The magnitude
        of $\Delta V$ is greater near each electrode and smaller near the
        midpoint. This occurs because the electric field is stronger close to
        the electrodes and weaker between them due to the superposition of two
        opposite charges.

  \item[Question 1.b:] What would be the appropriate trendline?

        The appropriate trendline is nonlinear and consistent with an
        inverse-square dependence. Since the electric field from each electrode
        follows $E \propto 1/r^2$, the combined field reflects dipole-like
        behavior rather than a linear relationship.

  \item[Question 1.c:] Because $\Delta x = 1\,\text{cm}$, how does $\Delta V$
        relate to the electric field?

        Since $E = -\frac{dV}{dx}$, and $\Delta x$ is constant, the electric
        field is approximately $E \approx -\frac{\Delta V}{\Delta x}$.
        Therefore, $\Delta V$ is directly proportional to the magnitude of the
        electric field.

  \item[Question 1.d:] What does the slope of the graph represent?

        The slope of the potential versus position graph represents the negative
        of the electric field. A steeper slope corresponds to a stronger
        electric field.

\end{description}

\subsection*{Activity 2}

\begin{figure}[H]
  \centering
  \includegraphics[width=0.6\textwidth]{Tables/table2.png}
  \caption{Table 2.}
\end{figure}

\begin{description}

\item[Question 2.a:] What is the difference from the previous results?

        Unlike the nonlinear behavior observed with point-like electrodes, the
        potential between parallel plate electrodes varies linearly with
        position. The electric field between the plates is approximately
        uniform.

\item[Question 2.b:] What would be the appropriate trendline?

        A linear trendline is appropriate. Since the electric field between
        parallel plates is constant, the potential varies proportionally with
        distance.

\end{description}

\subsection*{Activity 3}

\begin{figure}[H]
  \centering
  \includegraphics[width=0.6\textwidth]{Tables/table3.png}
  \caption{Table 3.}
\end{figure}

\begin{description}

\item[Question 3.a:] Compare your results to the two previous ones.

        In the cylindrical configuration, the electric field decreases with
        radial distance approximately as $E \propto 1/r$. The potential varies
        logarithmically with distance. This behavior differs from the
        inverse-square dependence of point charges and the uniform field of
        parallel plates. The geometry of the electrodes determines how the
        electric field varies in space.

\end{description}

\section{Above and Beyond}
% Key: Say how does this apply for real life, or give more historical/technical
% background, or anything not covered in class that relates to this specific lab
% topic.

% Must: A & B should touch at least one aspect of the lab topic. Please do not
% forget the citation if applicable (no required format).

% Strategy: Make it short: A & B does have to be "physics" in nature. So find
% some keywords: name(s) of the scientist(s), year of the inventions. You can
% also consider extra-work on the lab as an A & B.\usepackage{float}

Although this lab focuses on calculating electric fields from simple charge
distributions, the underlying mathematics is a powerful tool that extends far
beyond electrostatics. Using integrals to sum contributions from continuous
charges not only allows us to find the field of an infinite plane or a ring, but
also provides a foundation for more advanced topics such as electromagnetic
waves and gravitational fields. For example, the same principles of vector
addition and $1/r^2$ forces appear in Newton’s law of gravitation, while in wave
theory, superposition and field integration are essential for understanding
interference and diffraction patterns. This shows how a relatively simple
concept—summing the influence of many sources—can be applied across many areas
of physics.

\end{document}
