\documentclass[12pt]{article}
% Basic packages
\usepackage[utf8]{inputenc}   % UTF-8 encoding
\usepackage{geometry}         % Page margins
\usepackage{setspace}         % Double spacing
\usepackage{amsmath, amssymb} % Math symbols
\usepackage{graphicx}         % Include images
\usepackage{booktabs}         % Nice tables
\usepackage{enumitem}         % Custom lists

% Page setup: 1-inch margins
\geometry{
  letterpaper,
  margin=1in
}

% Double spacing
\doublespacing{}

% START OF DOCUMENT %
\begin{document}

\begin{center}
    {\textbf{Johnny A. Rojas \& Aiden Sitorus, George Nicacio}}\\
    {\Large \textbf{LAB REPORT: TITLE OF LAB REPORT}}
\end{center}

\section{Theory}
% Keys: khpContent "Introduction" and the "Introduction" at the beginning of the
% class.

% Must: Make sure address all equations and formulae used in the introductions
% (khpContent as well as in class).

% Strategy: Link these equations and formulae to each others in the order they
% were presented to you in class. Also, show how useful they are for  the lab.
% Drawing and sketches are allowed.

\section{Conclusion}
% Key: final results of each activities: slope, power, coefficient, etc. Do not
% report any graph or chart!!! was the experiment successful or not?

% Must:

% - Report the values or numbers that were MEASURED, their expected or known
%   values if applicable, and their absolute or/and percent errors, R^2,
%   Standard deviations, standard errors if applicable. Saying "the experiment was
%   successful with 5% error" is not acceptable because it misses the measured
%   value and the expected value. It should have been "Our measured power was
%   0.475 whereas the expected power is 0.5. The experiment was successful with 5%
%   error."

% - Discuss the kind of errors: systematic or random and explain. Here, you can
%   refer to the charts (without reporting the charts). You can say that most of
%   the measured values were below the expected for example.

% - List out the source or errors and be specific. Do not just say "the frictions
%   created errors", rather say "the frictions between the wheels and the axles
%   created errors".

% - Then, state the biggest source of errors and how it could be avoided.

% - Address the outliers if any, Also, explain why you rejected a data if you did.

% - You may suggest a way to improve the experiment. That will be always
%   appreciated.

% Strategy: do the above for each activity as one activity might be successful but
% the other had an issue, then end with an overall evaluation.

\section{Questions}
% Key: they are in khpContent

% Must: Itemize the answers. For example:

% - Activity #1, question 2-a

% - Activity #3, question 3

% Strategy: Most often, you find these questions in interrogative forms ending
% with a question mark "?" But sometimes the questions are in indirect forms so
% be watchful. Never write "See Excel file", just make a copy of the Excel file
% and report it again.

\section{Above and Beyond}
% Key: Say how does this apply for real life, or give more historical/technical
% background, or anything not covered in class that relates to this specific lab
% topic.

% Must: A & B should touch at least one aspect of the lab topic. Please do not
% forget the citation if applicable (no required format).

% Strategy: Make it short: A & B does have to be "physics" in nature. So find
% some keywords: name(s) of the scientist(s), year of the inventions. You can
% also consider extra-work on the lab as an A & B.

\end{document}
