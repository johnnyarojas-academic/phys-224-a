\documentclass[12pt]{article}

% Basic packages
\usepackage[utf8]{inputenc}   % UTF-8 encoding
\usepackage{geometry}         % Page margins
\usepackage{setspace}         % Double spacing
\usepackage{amsmath, amssymb} % Math symbols
\usepackage{graphicx}         % Include images
\usepackage{booktabs}         % Nice tables
\usepackage{enumitem}         % Custom lists

% Page setup: 1-inch margins
\geometry{
  letterpaper,
  margin=1in
}

% Double spacing
\doublespacing

% START OF DOCUMENT %
\begin{document}

\begin{center}
    {\textbf{Johnny A. Rojas \& Aiden Sitorus, George Nicacio}}\\
    {\Large \textbf{LAB REPORT: HUMAN EYE, A BIOPHYSICS LABORATORY}}
\end{center}

\section{Overarching Goal}
% Briefly state the main goal of the lab
The human eye can be described as a complex optical instrument that focuses
light rays using a muscular tissue (the eye lens) and a light-sensitive layer
that converts the focused light into electrical signals transmitted to the
brain.  The focus of this lab report is to utilize the knowledge of our previous
labs on geometriclens modeling to apply it at a fundamental level to generalize
the behaviour of the human eye and study is relation to common eye disseases.

\section{Theory}
% Explain the physics concepts behind the lab. Include relevant equations if needed.
As light travels through space, its interaction with matter affects the speed at
which the electromagnetic waves that carry it move through a medium. These
interactions produce refraction, a phenomenon that can be described at the
macroscopic scale by Snell’s law, $n_1sin(\theta_1)=n_2sin(\theta_2)$, where $n_1$
and $n_2$ are the indices of refraction of the two media, and $\theta_1$ and $\theta_2$
are the angles measured with respect to the normal to the surface. Rather than
tracking individual light–matter interactions, this description allows the
overall bending of light to be predicted using simple geometric relationships.

In the geometrical optics approximation, the propagation of light through
refracting surfaces is modeled using rays, which trace the direction of light
and provide an intuitive picture of how images are formed. This ray-based model
makes it possible to describe complex optical systems using a small set of
distances and angles. When applied to ideal lenses and restricted to small
angles relative to the optical axis, these relationships reduce to the thin-lens
equation:

$$ \frac{1}{f} = \frac{1}{d_o} + \frac{1}{d_i}  $$ where:
\begin{itemize}
  \item $f$ is the focal length (in $m$)
  \item $d_o$ is the object distance (in $m$)
  \item $d_i$ is the image distance (in $m$)
\end{itemize}

This equation provides a direct way to predict where an image will form once
the focal length of the lens is known.

The focal length itself depends on the physical properties of the lens. This
dependence is described by the lens maker’s equation,

$$ \frac{1}{f} = (n - 1) \left( \frac{1}{R_1} - \frac{1}{R_2} \right) $$ where:
\begin{itemize}
  \item $n$ is the index of refraction of the lens material and
  \item $R_1$ and $R_2$ are the radii of curvature of the lens surfaces.
\end{itemize}



Although highly simplified, together these model captures the essential behavior of many
real optical systems. In particular, it provides a useful framework for
understanding the human eye as an imaging system, where changes in lens shape
and focal length determine how light is brought into focus on the retina. In the
case of the human eye, the lens is flexible, and surrounding muscles can change
its curvature. By altering the radii of curvature, the eye effectively changes
its focal length, allowing objects at different distances to be brought into
focus on the retina.

\section{Apparatus}

\subsection{Homemade Eye Model}
Our setup for this section consisted of a ``screen'' (a plane where the light
was focused), a convex lens placed $20\,\text{cm}$ from the screen, and an
object placed $22.5\,\text{cm}$ from the convex lens. All measurements were
conducted in a dark environment to allow for greater precision.

\subsection{Variable Lens Model}

\subsubsection*{Fixed Lens Test}
Before beginning the experiment, we tested the PASCO eye model by placing a
$400\,\text{mm}$ focal length lens in the SEPTUM slot and positioning an object
$50\,\text{cm}$ from the lens.

\subsubsection*{Variable Lens}
Afterwards, the variable lens was filled with water until its membrane was flat.
The connecting tube was then attached to a syringe, which was filled
approximately three-quarters full. The tube was reconnected to the variable
lens, replacing the previous $400\,\text{mm}$ lens in the setup.


\section{Purpose of Each Activity}

\subsection*{Activity 1.}
To construct a simplified physical model of the human eye and establish its
correspondence with the geometrical optics framework introduced in the theory
section, identifying how the retina, lens, and optical axis are represented
within the thin-lens approximation.

\subsection*{Activity 2.}
To incorporate a variable-focus lens into the eye model in order to simulate
the eye’s ability to accommodate, directly applying the thin-lens equation to
demonstrate how changes in focal length affect image formation on the retina.

\subsection*{Activity 3.}
To quantitatively determine the magnitude of the focal length adjustment
required for proper retinal imaging, using the thin-lens equation to relate
object distance, image distance, and lens focal length in the context of the
human eye.

\subsection*{Activity 4.}
To analyze common eye diseases by interpreting them as deviations from the
idealized optical model, using the thin-lens and lens maker’s equations to
understand how changes in curvature or effective focal length lead to impaired
vision.

\section{Method}
% Step-by-step procedure
\begin{enumerate}
    \item Step 1
    \item Step 2
    \item Step 3
\end{enumerate}

\section{Data}
% Include independent, dependent, and control variables
\subsection{Variables and Hypothesis}
\begin{itemize}
    \item \textbf{Independent Variable:}
    \item \textbf{Dependent Variable:}
    \item \textbf{Control Variables:}
    \item \textbf{Hypothesis:}
\end{itemize}

\subsection{Measured Data}
% Example table
\begin{table}[h!]
\centering
\caption{Measured Data}
\begin{tabular}{lcc}
\toprule
Quantity & Value & Units \\
\midrule
Sample 1 & 0.00 & m \\
Sample 2 & 0.00 & m \\
% Add more rows as needed
\bottomrule
\end{tabular}
\end{table}

% Include graphs or figures if needed
%\begin{figure}[h!]
%\centering
%\includegraphics[width=0.7\textwidth]{graph.png}
%\caption{Caption describing the figure.}
%\end{figure}

\section{Calculations}
% Show your calculations, formulas, and any manipulations
\begin{align*}
% Example formula
y &= mx + b
\end{align*}

\section{Results}
% State final results with units and errors
\begin{itemize}
    \item Result 1: 0.00 units
    \item Result 2: 0.00 units
    \item Absolute/Percent Error: 0\%
\end{itemize}

\section{Conclusion \& Questions}
% Summarize whether the goal was achieved
% Discuss sources of error
\begin{itemize}
    \item Conclusion point 1
    \item Conclusion point 2
\end{itemize}

\subsection*{Questions}
\begin{enumerate}[label=\arabic*.]
    \item Answer to question 1
    \item Answer to question 2
\end{enumerate}

\section{Above and Beyond}
% Optional extra analysis, insights, or experiments

\end{document}
